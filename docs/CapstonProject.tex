\documentclass[12pt]{article} 
\usepackage{amsmath} 
\usepackage{titling}
\usepackage{graphicx}
\usepackage{hyperref}
\usepackage{navigator}
\usepackage[utf8]{inputenc}
\usepackage{indentfirst}
\usepackage{lettrine}
\usepackage[left=3cm,right=3cm]{geometry}
\newcommand{\subtitle}[1]{%
  \posttitle{%
    \par\end{center}
    \begin{center}\LARGE#1\end{center}
    \vskip0.5em}%
}

\title {\textbf{\Huge Applications of Generative Adversarial Networks in Hairstyle Transfer \vspace*{2\baselineskip}}} % Sets article title
\subtitle{\textbf{Capston Project Report}}
\author{
        \textbf{Student}\\Pham Cao Bang\\Ta Dang Khoa\\\\ \textbf{Instructor}\\Dr.Phan Duy Hung
 } % Sets authors name
\date{\vfill  \textbf{Bachelor of Computer Science\\
                     Hoa Lac Campus – FPT University \\ \today}
 } % Sets date for date compiled

% The preamble ends with the command \begin{document}
\begin{document} % All begin commands must be paired with an end command somewhere
    \maketitle % creates title using infromation in preamble (title, author, date)
    %\vspace*{19\baselineskip}
    \begin{center}
        \includegraphics[scale=1.5]{images/logo_fpt.png}\\
    \end{center}
    %Copyright page
    \vspace*{18\baselineskip}
    \begin{center}
     \Large Copyright $\copyright$ 2020 \\ 
            Pham Cao Bang \& Ta Dang Khoa \\ 
            All Rights Reserved
    \end{center}

    %Acknowledgement
    \newpage
    \begin{center}
        \section*{\Large ACKNOWLEDGEMENT}
    \end{center}
    \vspace*{2\baselineskip}
    
    Firstly, we would like to thank our instructor, Dr. Phan Duy Hung, for his patience and time, and for instructing and advising our enthusiastically. Secondly, we would also like to thank our University, FPT, for giving our a good environment to study and grow over the years. Finally, we always remember our family{\rq}s encouragement and support. We want to give them a special thanks, because they are the motivation for us to improve ourself every day.\\
    
    \newpage
    \begin{center}
        \section*{\Large ABSTRACT} % creates a section
    \end{center}
    \vspace*{2\baselineskip}  
	
	Recently, Generative Adversarial Networks (GANs) have been extensively studied, and a range of GANs architectures and related methods are proposed. The evolution of GANs make significant impact in the areas of computer vision, thus great advances have been made for insatance plausible image generation, image-to-image translation, facial attribute manipulation and related domains. Most of recently works have focused on a lot of domains instead of focused on single domain of GANs. Because of that, in this works, we want to focus on the applications in a single area named facial attribute manipulation, an interesting area of GANs. We apply the state of the arts architectures and methods of GANs in facial attribute manipulation for edditing hairstyle of given facial image.
	
    \vspace*{2\baselineskip}
    \noindent
    \textbf{Keywords:} Generative Adversarial Networks, GANs, Hairstyle Transfer, facial attribute manipulation.
    \newpage
    \section*{\Large TABLE OF CONTENTS}
    \newpage
    \section{\Large INTRODUCTION}
    \subsection{\Large Problem and motivation}
    
    
    \subsection{\Large Related works}
    \textbf{Generative Adversarial Networks (GANs)} are attracting a lot of researchers in deep learning community because of the challenges and interesting problems, which GANs provide. GANs have knowned as an area of computer vision, however, the applications of GANs have been applied to other domains such as natural language processing (NLP), time series synthesis, semantic segmentation, etc. GANs is a member of generative models family in machine learning. GANs offer advantages when compare to other generative model e.g., variational autoencoders such as an ability to handle sharp estimated density function, efficiently generating desired samples, eliminating deterministic bias and with good compatibility with the internal neural architectures. These properties of GANs help this networks success especially in the field of computer vision such as plausible image generation, image-to-image translation,  image super-resolution, facial attribute manipulation, etc.
    \\
    
    \textbf{Stylegan}
    \\
    
    \textbf{Interfacegan}
    
    \subsection{\Large Contribution}
    
    \section{\Large METHODOLOGY}
    \subsection{\Large Generative Adversarial Networks (GANs)}
    \subsection{\Large Stylegan}
    \subsection{\Large Interfacegan}
    \section{\Large IMPLEMENTATION AND ANALYSYS}
    \subsection{\Large Dataset}
    \subsection{\Large Experiment results}
    \section{\Large CONCLUSION AND PERSPECTIVES}
    
    
	
\end{document} % This is the end of the document

